\documentclass[]{ctexart}
\usepackage{amsmath}

\begin{document}

\section{Diffusion}

Diffusion模型定义了一个概率分布转换模型 $\mathcal{T}$, 能将原始数据 $x_0$ 构成的复杂分布

$$
\mathbf{x}_0 \sim p_{\text {complex }} \Longrightarrow \mathcal{T}\left(\mathbf{x}_0\right) \sim p_{\text {prior }}
$$

Diffusion模型提出可以用马尔科夫链(Markov Chain)来构造 $\mathcal{T}$, 定义一系列条件概率分布 $q\left(\mathbf{x}_t \mid \mathbf{x}_{t-1}\right), \quad t \in\{1,2,3 \ldots T\}$, 
将 $\mathbf{x}_0$ 依次转换为 $\mathbf{x}_1, \mathbf{x}_2, \ldots, \mathbf{x}_T$, 希望当 $T \rightarrow \inf$ 时, $\mathbf{x}_T \sim p_{\text {prior }}$

$$
\begin{gathered}
q\left(\mathbf{x}_t \mid \mathbf{x}_{t-1}\right)=\mathcal{N}\left(\mathbf{x}_t ; \sqrt{1-\beta_t} \mathbf{x}_{t-1}, \beta_t \mathrm{I}\right) \\
q\left(\mathbf{x}_T\right)=p_{\text {prior }}\left(\mathbf{x}_T\right)=\mathcal{N}\left(\mathbf{x}_T ; \mathbf{0}, \mathrm{I}\right) \quad \text { where } T \rightarrow \inf
\end{gathered}
$$

即已知 $\mathbf{x}_{t-1}$ 时, $\mathbf{x}_t$ 的概率分布是一个平均值为 $\sqrt{1-\beta_t} \mathbf{x}_{t-1}$ ,协方差为 $\beta_t \mathbf{I}$ 的正态分布。
其中$\beta$为扩散系数

根据数学技巧可得:(没有看的很懂推导的过程)

$$
\mathbf{x}_t=\sqrt{1-\beta_t} \mathbf{x}_{t-1}+\sqrt{\beta_t} \mathbf{z}_{t-1} \quad \text { where } \mathbf{z}_{t-1} \in \mathcal{N}(0, \mathbf{I})
$$

将$x_t$一层一层迭代,可以得到以下式子:

假设 $\alpha_t=1-\beta_t, \bar{\alpha}_t=\prod_{i=1}^T \alpha_i$, 那么:
$$
\begin{array}{rlr}
\mathbf{x}_t & =\sqrt{\alpha_t} \mathbf{x}_{t-1}+\sqrt{1-\alpha_t} \mathbf{z}_{t-1} & ; \text { where } \mathbf{z}_{t-1}, \mathbf{z}_{t-2}, \cdots \sim \mathcal{N}(\mathbf{0}, \mathbf{I}) \\
& =\sqrt{\alpha_t \alpha_{t-1}} \mathbf{x}_{t-2}+\sqrt{\alpha_t\left(1-\alpha_{t-1}\right)} \mathbf{z}_{t-2}+\sqrt{1-\alpha_t} \mathbf{z}_{t-1} & \\
& =\sqrt{\alpha_t \alpha_{t-1}} \mathbf{x}_{t-2}+\sqrt{1-\alpha_t \alpha_{t-1}} \overline{\mathbf{z}}_{t-2} & ; \text { where } \overline{\mathbf{z}}_{t-2}, \overline{\mathbf{z}}_{t-3}, \cdots \sim \mathcal{N}(\mathbf{0}, \mathbf{I})(3) \\
& =\ldots & \\
& =\sqrt{\bar{\alpha}_t} \mathbf{x}_0+\sqrt{1-\bar{\alpha}_t} \mathbf{z} &
\end{array}
$$

将公式(3)改写成条件概率的格式,可以得到:

$$
\begin{gathered}
q\left(\mathbf{x}_t \mid \mathbf{x}_{0}\right)=\mathcal{N}\left(\mathbf{x}_t ; \sqrt{\bar{\alpha_t}} \mathbf{x}_{t-1}, (1-\bar{\alpha_t}) \mathrm{I}\right) \\
\end{gathered}
$$

由于 $\beta_t \in(0,1)$, 那么 $\alpha_t \in(0,1)$ 。 $t \rightarrow \inf$ 时, $\bar{\alpha}_t \rightarrow 0$ 。可以看出, $\sqrt{1-\beta_t}$ 和 $\sqrt{\beta_t}$ 
作为系数保证了当 $T \rightarrow \inf$ 时, $q\left(\mathbf{x}_T\right)=p_{\text {prior }}\left(\mathbf{x}_T\right)=\mathbf{N}(0, \mathbf{I})$ 。
实际上,只要 $T$ 取一个足够大的值,不需要无限次迭代,得到的 分布就已经很接近于标准正态分布了。

下图为两个样本(红蓝两颜色)经过多次加噪最终得到$P_{prior}$

\section{逆Diffusion}
,$q\left(\mathbf{x}_{t-1} \mid \mathbf{x}_t\right)$ 分布未知
,但数学上可证连续扩散过程的逆转具有与正向过程相同的分布形式。即
即当扩散率 $\beta_t$ 足够小, 扩散次数足够多时,离散扩散过程接近于连续扩散过程, 
$q\left(\mathbf{x}_{t-1} \mid \mathbf{x}_t\right)$ 的分布形式同 $q\left(\mathbf{x}_t \mid \mathbf{x}_{t-1}\right)$ 一致, 同样是高斯分布。
    但是很难直接写出 $q\left(\mathbf{x}_{t-1} \mid \mathbf{x}_t\right)$ 的分布参数。为此, 可以用分布 $p_\theta\left(\mathbf{x}_{t-1} \mid \mathbf{x}_t\right)$ 来近似 $q\left(\mathbf{x}_{t-1} \mid \mathbf{x}_t\right)$ :

$$
p_\theta\left(\mathbf{x}_{t-1} \mid \mathbf{x}_t\right)=\mathcal{N}\left(\mathbf{x}_{t-1} ; \boldsymbol{\mu}_\theta\left(\mathbf{x}_t, t\right), \mathbf{\Sigma}_\theta\left(\mathbf{x}_t, t\right)\right)
$$

中 $\boldsymbol{\mu}_\theta$ 和 $\boldsymbol{\Sigma}_\theta$ 都是要学习的函数,接受 $\mathbf{x}_t, t$ 作为参数。
这样,连续迭代多次后,可以得到近似的真实数据分布 $p_\theta\left(\mathbf{x}_0\right)$ 为:
$$
p_\theta\left(\mathbf{x}_0\right)=\int p_\theta\left(\mathbf{x}_{0: T}\right) d \mathbf{x}_{1: T}
$$
其中 $p_\theta\left(\mathbf{x}_{0: T}\right)$ 为 $\mathbf{x}_0, \mathbf{x}_1, \cdots, \mathbf{x}_T$ 的联合概率分布。借助条件概率公式:
$$
p_\theta\left(\mathbf{x}_{0: T}\right)=p\left(\mathbf{x}_T\right) \prod_{t=1}^T p_\theta\left(\mathbf{x}_{t-1} \mid \mathbf{x}_t\right)
$$

\end{document}

